\chapter{Introduction}\label{chap:introduction}

\section{Motivation}
Krakow is one of the major Polish cities located in the Lesser Poland region in the southern part of the country. Due to its unfavourable topographical conditions (the city lies in the valley of the Vistula river) and the density of buildings reducing the air flow it struggles with rather severe air pollution.
\\\\
One of the more common pollutants that can be found in the city is particulate matter. In practice its concentrations are usually registered for its two types: PM2.5 and PM10, which denote airborne particles with diameters under 2.5 $\mu m$ and 10 $\mu m$ accordingly. In order to illustrate the scale of the problem it is worth noting that in the database published by the World Health Organisation in 2016 \cite{WHO2016AMBIENT} containing mean annual PM2.5 and PM10 concentrations in 1712 cities located in Europe (including Turkey), Krakow was ranked $55^{th} (37 \mu / m^3)$ and $74^{th} (51 \mu g / m^3)$, accordingly. Maximum mean annual concentrations recommended by the WHO are 10 $\mu g / m^3$ for PM2.5 and 20 $\mu g / m^3$ for PM10 \cite{WHO2008AQGUIDELINES}.
\\\\
The problem of poor air quality is especially prevalent during the heating season which spans roughly from the mid September to the beginning of April. During that period it is fairly common to observe in Krakow considerable exceedances of daily mean PM concentration limits. For example there are cases, when the maximum hourly PM10 concentrations reach levels above 200 $\mu g / m^3$ \cite{WIOSALERTS}, while the mean daily limit recommended by the Regional Inspectorate of Environmental Protection is 50 $\mu g / m^3$ \cite{WIOSNORMS}. One of the major causes of such high levels of pollution during winter is the fact that coal burning stoves are still popular among the residents of Krakow and nearby villages. According to the results of stocktaking of coal burners, commissioned by the city council and finished in 2015, there were approximately 24 thousand such stoves. It is assumed that since then more than 6 thousand of them have been liquidated \cite{COALBURNERS}, however the number of remaining ones is still considerable.
\\\\
Poor quality of air constitutes a serious problem for the residents of Krakow, given the harmful influence of pollutants on their health. According to the World Health Organisation prolonged exposure to high concentrations of particulate matter might cause increased morbidity from cardiovascular and respiratory diseases (e.g. asthma) as well as increased cardiopulmonary and lung cancer mortality and, as a result, reduced life expectancy \cite{WHO2013PMHEALTH}.
\\\\
Finding a reliable way of predicting PM2.5 concentrations in advance, while not solving the problem completely, could be beneficial for the residents, as it would allow to warn them about the incoming high pollution episodes. Potentially, it would be helpful not only for individuals but also for the local authorities, who could, for example, organise a day of free public transport in order to lower the production of exhaust fumes.

\section{Research goal} \label{sec:introduction-research-goal}
The problem of air pollution is common to many cities throughout the globe. Because of that, multiple attempts have been made to create predictive models which could be used for warning residents about the possible threats of high pollution levels, some of which were quite successful (for example \cite{VLACHOGIANNI20111559}, \cite{Chellali2016}, \cite{LI2017997}, \cite{SIWEK2016}). However, due to dependence on the local climate conditions, type of pollution and availability of historical data, findings reported in a specific study might not be directly applicable to other locations. Another problem is the diversity of forecasting goals. There have been already at least two similar studies conducted for Krakow -  \cite{LOZOWICKA2005} and \cite{Pawul2016} - however they were focused on daily aggregated variables - a sum of mean daily concentrations of a few pollutants divided by their limits and mean daily $PM10$. So far, it seems, there have been no local studies devoted to more fine-grained, hourly forecasts, which could provide an estimation of the changes in pollution levels throughout the next day. The goal of this study is to test if a few selected statistical forecast models - multiple linear regression, support vector regression and artificial neural networks - are viable for such a task - prediction of mean hourly PM2.5 concentrations in Krakow 24 hours in advance. The process is comprised of a few steps: data gathering and preparation, model creation, hyperparameter tuning and verification of the results based on statistical measures.

\section{Structure of the thesis}
This document is divided into two parts (excluding this introductory one). The first one  is focused on providing a summary of results reported in similar studies conducted in different cities for various pollutants and forecast types (chapter \ref{chap:related-work}). It is organised in sections dedicated to specific predictive models. Each section starts with a theoretical overview of a given model and proceeds to discussing examples of research which it was applied to.
\\\\
The second part contains a presentation of the contribution of this thesis. Firstly, the testing procedure is set forth (chapter \ref{chap:methodology}). Then, a description of the data set is provided, which concerns the data sources, collected variables, their statistics and relationships between them (chapter \ref{chap:dataset}). It is followed by a discussion about the results of the performed experiments, which is meant to verify the research goal presented in section \ref{sec:introduction-research-goal} (chapter \ref{chap:results}). Lastly, conclusions drawn from the tests are presented. Additionally, some ways of extending the research are proposed (chapter \ref{chap:conclusions}).
