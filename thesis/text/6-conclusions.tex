\chapter{Conclusions and future works} \label{chap:conclusions}

The goal of this study was to investigate the applicability of three statistical forecasting models: multiple linear regression (MLR), support vector regression (SVR) and a multi-layered perceptron (MLP), to the task of predicting mean hourly PM2.5 concentrations in Krakow 24 hours in advance. The forecasts were made based on the historical pollution levels and values of a few meteorological and temporal variables. The tests were performed for three monitoring stations operated by the Regional Inspectorate of Environmental Protection.
\\\\
In most cases the type of the best model was found to depend on the season: linear regression performed best for winter, SVR and MLPs for spring, MLR and SVR for summer. For autumn no single model was clearly performing best. The results vary depending on the season and station, which the data come from: for winter RMSE ranges from 43.912 to 55.634 $\mu g/m^3$, for spring - from 14.408 to 16.306 $\mu g/m^3$, for summer from 6.855 to 8.856 and for autumn from 21.768 to 25.870 $\mu g/m^3$. The results suggest that the tested models may be potentially useful, however they were found to have two rather important drawbacks:
\begin{itemize}
    \item they are incapable of predicting spikes of the PM2.5 concentrations with a satisfactory accuracy (in the specific configuration used in the study),
    \item forecasts seem to be delayed relative to the actual changes in the pollution levels.
\end{itemize}
In order to reduce the severity of these shortcomings, further research is encouraged. Some of the possible directions of study include:
\begin{itemize}
    \item using a larger data set (observations taken before 2014),
    \item including additional variables e.g. traffic intensity,
    \item reducing the number of the input variables,
    \item investigating alternative forecasting models, for example: ARIMA models, recurrent neural networks (long short-term memory neural networks).
\end{itemize}